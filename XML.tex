\section{Header der XML Dateien}\label{xml:header}
\begin{minted}{xml}
    <?xml version="1.0" encoding="UTF-8" standalone="no"?>
\end{minted}

Diese Zeile stellt die erste Infomation dar, die jede XML Datei enthalten soll. An dieser Stelle sind sich einige
Quellen im Netz uneinig. W3Schools beispielsweise lässt zu, eines oder mehrere der drei Attribute auszulassen. XMLParse
und XMLlint verweigern dann allerdings die Validierung. Nur das Standalone Attribut wird schlichtweg ignoriert. Da
XMLParse und XMLline die Standardbibliotheken zur Validierung von XML/DTD/Schema verwenden, sollte in der Praxis so
gearbeitet werden, dass beide Parser validieren.
\subsection{Version}
Das \textit{version} Attribut ist in 99\% aller Fälle die ''1.0''. Das liegt daran, dass es zwar eine Version ''1.1''
gibt, diese jedoch in der Praxis nicht verwendet wird. Grund dafür ist primär, dass die meisten Parser die Version
''1.1'' nicht verstehen. Die Versionierungist jedoch irreführend. XML ''1.0'' liegt aktuell in der \textit{fünften
Auflage} vor. Eine Bezeichnung wie ''1.0-5'' wäre hier sinnvoller, ist aber nicht umgesetzt worden.

\subsection{encoding}
Mit dem Attribut encoding wird angegeben, welche Zeichenkodierung zum Speichern der XML-Datei verwendet werden soll. Folgende
Angaben sollte jeder XML-Parser kennen.
\begin{longtable}{|l|p{9cm}|}\hline
  encoding=''UTF-8'' & internationaler Kodierung auf Basis der ISO/IEC-10646-Norm mit mindestens 8 Bit Zeichenbreite
  \\\hline
  encoding=''UTF-16'' & internationale Kodierung auf Basis der ISO/IEC-10646-Norm mit mindestens 16 Bit Zeichenbreite
  \\\hline
  encoding=''ISO-8859-?'' & ISO-Kodierung für diverse Sprachen mit ? als 1 für Westeuropa, 2 für Osteuropa, usw.. 
  \\\hline
\end{longtable}
Ohne Angabe des encoding wird in der Regel UTF-8 angenommen. Hier aber nochmals der Hinweis, dass XMLlint in der
aktuellen Version\footnote{using libxml version 20903} nicht glücklich ist, sollte das encoding ausgelassen werden.

\subsection{standalone}
Mit dem \textit{standalone} Attribut kann dem Parser bereits vor dem eigentlichen parsen bekannt gemacht werden, ob das
Dokument alleinstehend ist, oder ob der Parser eine externe Resource (z.B. eine DTD) mit berücksichtigen muss.

\subsection{Weitere Verarbeitungsanweisungen}
Hin und wieder kann es notwendig sein, dass auch innerhalb der XML-Datei spezielle Anweisungen für die auslesende
Software notiert werden müssen. Solche Anweisungen nennt man Verarbeitungsanweisungen (processing instructions).

Diese Anweisungen sind frei erfunden und leben nicht in einer XML-spezifischen Welt, sondern in der Welt des parsenden
Programmes. So kann z.B. einem parsenden Programm ein hinweis gegeben werden, dass sich an dieser Stelle ein besonderer
Datensatzteil befindet.

Diese Anweisungen werden ähnlich zum Header mit \textless?foo anweisung ?\textgreater

\section{Doctypeanweisung}\label{xml:doctype}
Direkt nach dem Header folgt die Angabe des Dokumententyps. 
Die Dokumenttyp-Deklaration dient dazu, den Bezug zu einer Document Type Definition (DTD) herzustellen. Das Thema DTD
wird im Abschnitt \ref{DTD}
\begin{listing}[H]
\begin{minted}{dtd}
  <!DOCTYPE Gruss [
    <!ELEMENT Gruss (#PCDATA)>
  ]>
  ...
\end{minted}
\caption{Beispiel einer Dokumenttypdeklaration mit interner DTD}
\end{listing}
Die Dokumenttyp-Deklaration beginnt mit \textless!DOCTYPE. Dahinter folgt, durch Leerraum getrennt, der Name des Dokumenttyps - im
Beispiel Gruss. Der Name des Dokumenttyps muss mit dem Namen des Dokument-Elements, also dem äußersten Element in der
XML Datei, identisch sein.

Eine interne DTD folgt direkt auf das Dokument-Element. Für eine externe DTD sieht ein Beispiel wie folgt aus:
\begin{listing}[H]
\begin{minted}{dtd}
  <!DOCTYPE Gruss SYSTEM "Gruss.dtd">
  ...
\end{minted}
\caption{Beispiel einer Dokumenttypdeklaration mit externer DTD}
\end{listing}

Hier wird dem Parser mitgeteilt wo sich die DTD befindet. Dabei kann eine lokale oder eine fremde URL angegeben werden.
Zusätzlich dazu kann statt dem Wort ''SYSTEM'' auch das Wort ''PUBLIC'' verwendet werden. Öffentliche Bezeichner sind
für den Fall gedacht, dass der verarbeitende XML-Parser keine Möglichkeit hat, auf die entfernte DTD zuzugreifen (z.B.
weil keine Internet-Verbindung besteht). Eine echte Gültigkeitsprüfung gegen die DTD ist in einem solchen Fall natürlich
nur möglich, wenn der Parser die Regeln der DTD kennt.

\begin{listing}[H]
\begin{minted}{dtd}
  <!DOCTYPE Gruss PUBLIC "-//MeineFirma Solutions//DTD EMail V 1.0//DE" 
  "http://www.example.org/dtds/Gruss.dtd">
  ...
\end{minted}
\caption{Beispiel einer Dokumenttypdeklaration mit externer DTD}
\end{listing}

Öffentliche Bezeichner bestehen aus drei Teilen: erstens der Angabe zur veröffentlichenden Person oder Institution,
zweitens einer Bezeichnung der DTD, beginnend mit DTD, gefolgt vom Namen der DTD und einer Versionsangabe, und drittens
aus einer Angabe zur Sprache, in der die Namen der Elemente, Attribute usw. in der DTD definiert sind. DE steht für
deutsch, EN für englisch usw. Hinter der Angabe zum öffentlichen Bezeichner kann dann noch - wie im obigen Beispiel -
die Internet-Adresse mit dem tatsächlichen Speicherort der DTD folgen.
XMLlint kann diese DTD z.B. vor dem validieren herunterladen und verarbeiten (sofern dies möglich ist).

\section{XML Elemente}
Jedes Dokument hat genau ein Element. Das klingt zunächst etwas merkwürdig, ist aber korrekt. Das Element nennt sich
Root Element und kann seinerseits wieder Elemente enthalten. Diese Schachtelung führt immer zu einer baumartigen
Struktur. Diese Struktur ist gewollt, da sie das maschinelle Verarbeiten erleichtert.
\subsection{Tags}
Alle Elemente werden durch Tags eingeleitet und ausgeleitet. Das Wort Tag ist dem englischen Wort für ''Etikett''
entliehen und könnte auf Deutsch auch zu ''Abgrenzer'' übersetzt werden.
Tags werden wie bei HTML Elementen verwendet. Ein Tag wird mit \textless FOO\textgreater eingeleitet und mit
\textless/FOO\textgreater ausgeleitet.

\begin{minted}{xml}
<FOO>Lorem Ipsim</FOO>
\end{minted}

Abgrenzend zu HTML sollte hier erwähnt werden, dass die Benennung der Tags zwar frei wählbar ist, allerdings vier
kleinen Regeln folgen müssen: \\
Sie dürfen nicht
\begin{itemize}
    \item mit der Zeichenfolge xml 
    \item mit Ziffern beginnen
    \item Gleichheitszeichen 
    \item Leerzeichen enthalten
\end{itemize}

Besonderheit: Leere Elemente. Für Leere Elemente gibt es die Kurzschreibweise:
 \begin{minted}{xml}
<FOO/>
\end{minted}
 
\subsection{Attribute}
Elemente können (falls nötig) mit Attributen versehen werden. Attribute sind in der Regel eine schlechte Idee, werden
aber aus unerfindlichen Gründen zugelassen und führen das Grundkonzept von XML, nämlich das X\footnote{XML: Extensible
Markup Language}, ad absurdum. Attribute können eben nicht weiter verschachtelt werden und sind somit nicht
erweiterbar. Hingegen kann einem Element jederzeit eine weitere Verschachtelungsebene hinzugefügt werden. Das Portieren
der bereits vorhandenen Dokumente kann allerdings auch dabei recht komplex werden. Daher sollte man das X eher auf das
Designen seiner Datenstrukturen beziehen. In den wenigen Fällen, in denen klar ist, dass es keine Erweiterbarkeit gibt,
kann es beim Parsen etwas leichter werden, wenn Attribute verwendet werden. Beispielsweise kann eine ID als Attribut
vergeben werden, die ein Element (bzw. einen kompletten Subtree in der Baumstruktur) kennzeichnet. So kann beim Parsen
explizit (ähnlich zu HTML) auf ein Element zugegriffen werden, und der Subtree bei allen nicht matchenden Elementen
ignoriert werden.

Um eine abstrakte Faustregel an dieser Stelle zu liefern kann formuliert werden:
\framebox{Es ist sinnvoll Metadaten als Attribute zu setzen, Daten selbst jedoch als Elemente.}

Attribute müssen diesen Regeln folgen:
\begin{itemize}
    \item Ein Attributname darf nur einmal in einem Tag vorkommen.
    \item Der Typ des Attributs muss in der DTD definiert sein.
    \item Attribute dürfen nicht auf externe Entitäten verweisen (siehe Abschnitt \ref{XML:Entities}).
    \item Durch eine Entität ersetzte Inhalte dürfen das Zeichen \textless\ nicht enthalten.
\end{itemize}

\begin{listing}[H]
\begin{minted}{xml}
  <table cellpadding="5" cellspacing="1">
    <tr>
      <td>Eine Zelle</td>
    </tr>
  </table>
\end{minted}
\caption{Beispiel einer Mixtur aus Attributen und Elementen in HTML}
\end{listing}

Im Beispiel wurde die Faustregel eingehalten. Daten über die Tabelle sind als Attribute geschrieben worden, die
eigentlichen Daten als Element.

Auch Attribute sind in XML entgegen zu HTML Case-Sensitive.

\subsection{Entitäten}\label{XML:Entities}
Entitäten können in XML grob als "Platzhalter" mit Inhalt verstanden werden. Eine Entity wird in der DTD definiert.
\begin{listing}[H]
\begin{minted}{dtd}
<!ENTITY writer "Donald Duck.">
\end{minted}
\caption{Beispiel einer lokalen DTD mit interner Entity}
\end{listing}

Entitäten können auch extern eingelesen werden:
\begin{listing}[H]
\begin{minted}{dtd}
<!ENTITY writer SYSTEM "http://www.w3schools.com/entities.dtd">
\end{minted}
\caption{Beispiel einer lokalen DTD mit externer Entity}
\end{listing}

Hier muss der Parser nun die Entität extern auflösen. Attribute können auf diese externen Definitionen nicht
zugegreifen.

\subsection{Kommentare}
Kommentare werden wiein HTML mit \textless!-- eingeleitet und mit --\textgreater ausgeleitet. Kommentare können in DTD bereichen und XML
Bereichen verwendet werden.
\begin{minted}{xml}
<!-- Ich bin nur ein Kommentar -->
\end{minted}

\subsection{Namensraum}
Namensräume werden in XML ebenfalls unterstützt. Jedoch ist die Unterstützung leider etwas eingeschränkt.
Ein benannter Namensraum lässt sich leider nicht global über Elemente legen, sondern wird über einen Präfix definiert, der jedem
zugehörigen Element vergeben werden muss. Sowohl in der DTD als auch im Dokument selbst.
Um einen Namensraum zu definieren erhält ein Dokument das Attribut xmlns:
\begin{minted}{xml}
<note xmlns:foo="http://someUrl.de">...
\end{minted}

Dabei stellt foo den Namen des Namensraums dar. Alle zugehörigen Kinder von note können jetzt mit dem Präfix foo versehen und
dem Namensraum zugewiesen werden.

\begin{minted}{xml}
<note xmlns:foo="http://someUrl.de">
  <foo:bar>quux</foo:bar>
</note>
\end{minted}

Es können beliebig viele Namensräume kombiniert werden. Die eingangs erwähnte schwache Unterstützung der Namensräume
liegt daran, dass der Präfix immer verwendet werden muss, auch wenn es nur einen Namespace gibt. Um diese Schwäche etwas
abzufedern lässt sich auch ein Default-Namensraum ohne Präfix definieren, so dass die Kind-Elemente nicht alle manuell
gepräfixed werden müsswen. Dazu lässt man einfach die Benennung des xmlns Attributs aus:

\begin{minted}{xml}
<note xmlns"http://someUrl.de">...</note>
\end{minted}

Es ist nun möglich innerhalb von \textless note\textgreater den Default-Namensraum zu ändern. 
\begin{minted}{xml}
<note xmlns="http://someUrl.de">
  <foo>quux</foo>
  <bar xmlns="http://someOtherUrl.de">
    <more>...</more>
  </bar>
</note>
\end{minted}

Die Elemente note und foo gehören zum Namensraum ''http://someUrl.de'', bar und more gehören zu
''http://someOtherUrl.de''.

Jedes xmlns Attribut muss wie bereits erwähnt in der DTD bekannt gemacht werden. Siehe dazu Sektion \ref{DTD:attribute}.

\subsection{Sonderzeichen}
Auch in XML haben wie in HTML einige Zeichen eine besondere Bedeutung. Folgende Zeichen haben vordefinierte Entities:
\begin{longtable}{|l|l|}\hline
  Zeichen & Entity \\\hline
  $<$ & \&lt; \\\hline
  $>$ & \&gt; \\\hline
  \& & \&amp; \\\hline
  '' & \&quot; \\\hline
  ' & \&apos; \\\hline
\end{longtable}

\section{DTDs}\label{DTD}
Document Type Definitions sind die erste Stufe nicht nur wohlgeformtes XML (syntaktisch korrekt) zu erzwingen, sondern
auch Regeln für die Elemente und Attribute zu erstellen. DTDs sind allerdings in Ihrer Form beschränkt. Die zweite
Stufe, das XML Schema, wird in Abschnitt \ref{Schema} beschrieben.

DTDs können intern oder extern geschrieben werden. Wie in den vorhergehenden Beispielen bereits gesehen werden konnte
werden interne DTDs direkt nach der DOCTYPE Definition eingebunden, externe über das Schlüsselwort SYSTEM.

\begin{listing}[H]
\begin{minted}{dtd}
  <!DOCTYPE Gruss SYSTEM "Gruss.dtd">
  ...
\end{minted}
\caption{Beispiel einer Dokumenttypdeklaration mit externer DTD}
\end{listing}
\begin{listing}[H]
\begin{minted}{dtd}
  <!DOCTYPE Gruss [ ***DTD*** ]>
  ...
\end{minted}
\caption{Beispiel einer Dokumenttypdeklaration mit interner DTD}
\end{listing}

\subsection{Elemente}
In der DTD können Elemente in Ihrer Reihenfolge und Beschaffenheit definiert werden. Eingeleitet werden DTD Elemente mit
\textless!ELEMENT\textgreater. 

\begin{listing}[H]
\begin{minted}{dtd}
  <!ELEMENT name inhalt>
\end{minted}
\caption{Syntax für die Deklaration von Elementen}
\end{listing}

Der Elementname entspricht dem Tag Namen und der Inhalt wird kann aus folgenden Bausteinen bestehen:
\begin{longtable}{|p{0.25\textwidth}|p{0.7\textwidth}|}\hline
  EMPTY & Das Element ist immer leer\\\hline
  ANY & Es kann jeder beliebeige Inhalt (Text, Elemente) enthalten sein\\\hline
  (\#PCDATA) & Es darf nur Text (Parsable Character Data) enthalten sein\\\hline
  (el1| el2,el3* | el4+ ....) & Angabe der möglichen Elemente (getrennt durch |-Symbol) oder der genauen Reihenfolge
  (getrennt durch Komma). Zusammenfassen durch runde Klammern ist erlaubt. Die Sonderzeichen nach den Elementnamen sind:
  \begin{tabular}{|l|l|}\hline
    Keines & Element kommt genau einmal vor \\\hline
    + & Element kommt mindestens einmal vor \\\hline
    * & Element kommt null bis n mal vor \\\hline
    ? & Element kommt genau 0 oder 1 mal vor.
  \end{tabular}\\\hline
\end{longtable}
Zu beachten sind hier die Klammern.

\begin{listing}[H]
\begin{minted}{dtd}
<!ELEMENT name (meinelement, 
  deinelement+, 
  irgendeinelement|einandereselement)
>
\end{minted}
\caption{Beispiel für die Deklaration von Elementen}
\end{listing}


\subsection{Attribute}\label{DTD:attribute}

Attribute werden in einer DTD sehr ähnlich angegegeben:
\begin{listing}[H]
\begin{minted}{dtd}
  <!ATTLIST Elementname Attributname1 inhalt [TYP oder Wert] 
                        Attributname2 inhalt [Typ oder Wert] ...>
\end{minted}
\caption{Syntax für die Deklaration von Attributen}
\end{listing}

Der Inhalt ist in der Regel CDATA, kann aber auch ein Definierter Enum sein, also eine Angabe von möglichen Werten. Dazu
ein etwas komplexeres Beispiel:
\begin{listing}[H]
\begin{minted}{dtd}
<!ELEMENT hotels (hotel)*>
<!ELEMENT hotel (#PCDATA)>
<!ATTLIST hotel
  name            CDATA               #REQUIRED
  klasse          (I|II|III|IV|V)     #FIXED "II"
  einzelzimmer    (ja|nein)           #IMPLIED
  doppelzimmer    (ja|nein)           "ja"
>
\end{minted}
\caption{Beispiel für die Deklaration von Attributen}
\end{listing}

Für CDATA darf der Inhalt ''irgendeine'' Abfolge von Zeichen sein. Für Klasse dürfen nur Römische Zahlen von 1-5 angegeben
werden. Einzel- und Doppelzimmer können ja und nein enthalten. Zusätzlich sehen wir die Typworte \#REQUIRED und
\#IMPLIED und \#FIXED, sowie einen Wert des vorstehenden Enums direkt. 

\subsubsection{\#IMPLIED}
Das Schlüsselwort \#IMPLIED ist hier etwas ungünstig, denn es impliziert nicht, sondern sagt lediglich aus, das der Wert
Optional ist, also auch ausgelassen werden darf. Da Attribute nicht mehrfach verwendet werden dürfen entspricht \#IMPLIED
also dem Elementzusatz ? (0 oder 1 Vorkommen). Warum XML hier einer anderen Notation folgt ist mir leider nicht
ersichtlich.

\subsubsection{\#REQUIRED}
Das Schlüsselwort \#REQUIRED sagt aus, dass das Attribut angegeben sein muss. Es entspricht dem Zusatzlosen Element. Es
muss also genau einmal vorkommen.

\subsubsection{\#FIXED}
Etwas verwirrender ist das \#FIXED Schlüsselwort, denn es setzt dem Attribut einen festen Wert von dem nicht abgewichen
werden darf. Zudem muss dieser Wert im Datensatz angegeben werden (darf nicht ausgelassen werden). So kann
Beispielsweise ein Namensraum erzwungen werden. Ein tieferer Sinn erschließt sich mir auch hier nicht.

\subsubsection{Vorgabewerte}
Wird das Attribut Doppelzimmer ausgelassen, so wird im obigen Beispiel einfach der Wert ''ja'' angenommen.

\subsubsection{Weitere nutzbare Typen}
Statt den angegeben Typen CDATA und Enum können noch die Typen IDREF, ID und ENTITY. Auf letztere werde ich an dieser
Stelle nicht eingehen, da das Einbinden von Entitäten in DTDs kompliziert und teilweise unzulässig ist. IDs und IDREFs
verhalten sich in der Theorie wie Primärschlüssel und Fremdschlüssel bei relationalen Datenbanken. Leider ist die
Prüfung oft nicht restriktiv umgesetzt. Eine ID muss eindeutig im gesamten Dokument sein, eine IDREF muss auf eine
andere ID verweisen (Elternelement). Feinheiten wie zusammengesetzte Schlüssel gibt es bei DTDs leider nicht.


\section{XML Schema}\label{Schema}
XML Schema sind, wie bereits erwähnt, der inoffizielle Nachfolger von DTDs. XML Schema wird auch als XSD (XML Schema
Definition) bezeichnet und will den gleichen Bereich abdecken wie die DTD und Mängel an den DTDs ausgleichen. Einer der
größten Neuerungen ist die Unterstützung von Datentypen. So ist es einfacher Datentypen zu erzwingen und sogar selbst zu
definieren. Man stelle sich vor, man würde alle Buchstaben, außer einem ''d'' als Enum in einer DTD definieren müssen.
XSD bietet diese Möglichkeit.

\subsection{Header}
XSD Dateien werden in XML geschrieben. Im Gegensatz zu den DTDs beginnen wir also wieder mit dem 
\begin{minted}{xml}
<?xml version="1.0" encoding="UTF-8"?>
\end{minted}
Header als Einleitung. Anschließend verwenden wir einen besonderen Namensraum: Der xmlns:xs-Namensraum muss für XSD
immer folgendermaßend efiniert werden.
\begin{minted}{xml}
<xs:schema xmlns:xs="http://www.w3.org/2001/XMLSchema"
...
\end{minted}
Dabei ist der Bezeichner \textit{xs} beliebig gewählt werden. Dieser Namensraum beschreibt die vordefinierten Datentypen. 
Zusätzlich können die Attribute xmlns und targetNamespace gesetzt werden:
\begin{minted}{xml}
<xs:schema xmlns:xs="http://www.w3.org/2001/XMLSchema"
  xmlns="http://meinnamespace.meinefirma.de"
  targetNamespace="http://meinnamespace.meinefirma.de"
>
\end{minted}
Dabei stellt der targetNamespace den Namespace der gegen das Schema zu validierenden XML Datei dar und der xmlns den
Default Namensraum des Schemas selbst.

Will man ein so eingeleitetes Schema verwenden so würde die XML Datei wie folgt aussehen:
\begin{minted}{xml}
<?xml version="1.0" encoding="ISO-8859-1"?>
  <MeinRootElement xmlns="http://meinnamespace.meinefirma.de"
    xmlns:xsi="http://www.w3.org/2001/XMLSchema-instance"
    xsi:schemaLocation="http://meinnamespace.meinefirma.de
    http://www.meinefirma.de/MeinSchema.xsd">
    <MeinUnterElement>
      Elementinhalt
    </MeinUnterElement>
  </MeinRootElement>
\end{minted}

Auch der Namensraum xsi ist hier vorgegeben. So wird dem Parser mitgeteilt, dass es sich um eine Instanz eines mit einer
XSD Datei zu validierenden Schemas handelt.

Anschließend wird das Schema mit xsi:schemaLocation bekannt gemacht. Dabei wird der Schema Namensraum mit Leerzeichen
getrennt vor die URI gestellt.

\subsection{Elemente}
Wir bleiben beim Schemanamensraum xs wie im obigen Beispiel. Um ein Element zu definieren müssen wir nun folgermaßen
vorgehen:
\begin{minted}{xml}
<xs:element name="artikel">...</xs:element>
\end{minted}
Wir öffnen also ein element aus dem W3C Namensraum, welches den Bezeichner ''artikel'' trägt.
Innerhalb des Tags bestimmen wir nun den Inhalt des Elements.
\begin{minted}{xml}
  <xs:element name="artikel">
    <xs:complexType>
      <xs:sequence>
        <xs:element name="titel" type="xs:string"/>
        <xs:element name="teaser" type="xs:string"/>
        <xs:element name="inhalt" type="xs:string"/>
        <xs:element name="wertung" type="xs:decimal"/>
        <xs:element name="anzahl" type="xs:integer"/>
        <xs:element name="oem" type="xs:boolean"/>
        <xs:element name="daum" type="xs:date"/>
        <xs:element name="zeit" type="xs:time"/>
      </xs:sequence>
    </xs:complexType>
  </xs:element>
\end{minted}

An diesem Beispiel sehen wir, dass wir zunächst einen neuen komplexen Datentypen definieren müssen, welcher aus einer
Sequenz von Elementen besteht. Die einzelnen Elemente haben hier den Datentyp angegeben.

\subsubsection{Mehrfach vorkommende Elemente und beliebeige Reihenfolge}
Das oben angegebene Beispiel setzt eine feste Reihenfolge mit genau einem Vorkommen jedes Elenents fest.
Um Elemente mit den DTD zusätzen ?+* zu versehen gibt es im  XSD die Attribute ''minOccurs'' und ''maxOccurs''.
Diese Attribute nehmen Zahlenwerte oder ''unbounded'' an.

Beispiel:
\begin{minted}{xml}
  <xs:element name="anzahl" 
              type="xs:integer" 
              minOccurs="0" 
              maxOccurs="unbounded"/>
\end{minted}
Hier können beliebig viele Elemente <Anzahl> vorkommen. Auch hier zeigt sich eine feinere Restriktion als bei DTDs, so
kann auch 5 bis 9 Mal definiert werden, was zuvor nicht möglich war.

Um die Reihenfolge beliebig zu halten kann statt xs:sequence einfach der Tag xs:all verwendet werden. 
\subsection{Attribute}
Dieser Abschnitt ist kurz. Attribute sind immer einfache Typen und folgen also den nicht-komplexen Elementregeln.
\begin{minted}{xml}
<xs:attribute name="inhalt" type="xs:string"/>
\end{minted}

Die Werte default, fixed, optional (statt \#IMPLIED) und required können mit dem \textit{use} Attribut angegeben werden:
\begin{minted}{xml}
<xs:attribute name="inhalt" type="xs:string" use="required"/>
\end{minted}


